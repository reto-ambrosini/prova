\documentclass[a4paper]{scrartcl}
\usepackage[T1]{fontenc}
\usepackage[italian]{babel}
\usepackage{lmodern}
\usepackage{biblatex}
\usepackage{csquotes}
\addbibresource{copernico-bib.bib}
% con PdfLatex
%\usepackage[utf8]{inputenc}

% con XeLatex
\usepackage{fontspec}
%\setmainfont{OpenDyslexic}	% font di sistema
\setmainfont{DejaVu Sans}

\usepackage{verbatim}	% permette di commentare con \begin{comment}
\usepackage{multicol}	% tabella con più colonne
\usepackage{graphicx}	
\usepackage{chronosys}

% pacchetti matematici
\usepackage{siunitx}	% unità di misura
\usepackage{amsmath}
\usepackage{amsfonts}
\usepackage{amsthm}
\usepackage{amssymb}
\usepackage[makeroom]{cancel} % per poter stralciare espressioni matematiche

\begin{document}

\author{Reto Ambrosini \and Luca Moscatelli}
\title{La rivoluzione copernicana}

\maketitle

\tableofcontents

\newpage

\section{Tolomeo}

Claudio Tolomeo è nato a Pelusio, Egitto verso il 100 ed è morto ad Alessandria d'Egitto verso il 175. È stato un matematico, astronomo e geografo greco. È l'autore dell'almagesto, un'importante opera astronomica scritta verso il 150 che ha costituito la base delle conoscenze astronomiche per più di mille anni. Il titolo originale greco dell'opera era "Mathematiké sýntaxis" (Trattato matematico).
Si fa riferimento al sistema tolemaico intendendo il sistema geocentrico da lui messo a punto.

\section{Copernico}

\begin{quote}
	In medio vero omnium residet sol. \autocite[98]{copernico:derevolutionibus}\\ 
	In mezzo a tutti sta il sole.
\end{quote}


Mikołaj Kopernik (Niccolò Copernico in italiano, Nicolaus Copernicus in latino) è nato a Toruń, Polonia, il 19 febbraio 1473 ed è morto a Frombork, Polonia, il 24 maggio 1543).

\subsection{Commentariolus}

Verso il 1509 Copernico scrive un libricino intitolato Commentariolus, si tratta di una prima esposizione della sua teoria eliocentrica, una specie di brutta copia del De Revolutionibus, che non è ancora pronto per la pubblicazione. Questo libretto viene stampato in pochissimi esemplari, distribuiti solo ad alcuni amici dell'autore.

Il Commentariolus getta le basi della teoria eliocentrica copernicana, in particolare propone 7 assiomi (o petitiones) che precisano le caratteristiche del suo nuovo sistema.

\begin{enumerate}
	\item Non esiste un solo centro di tutti gli orbi celesti o sfere.
	
	\item Il centro della Terra non è il centro dell'Universo, ma solo della gravità e della sfera della Luna.
	
	\item Tutte le sfere ruotano intorno al Sole come al loro punto centrale e pertanto il centro dell'Universo è intorno al Sole.
	
	\item Il rapporto fra la distanza della Terra dal Sole e l'altezza del firmamento è minore del rapporto fra il raggio terrestre e la distanza Terra-Sole, di modo che la distanza della Terra dal Sole è impercettibile in confronto all'altezza del firmamento.
	
	\item Qualunque moto appaia nel firmamento, non deriva da un qualche moto del firmamento, ma dal moto della Terra. Pertanto la Terra, con gli elementi a lei piú vicini [le acque che giacciono sulla sua superficie e l'atmosfera] compie una completa rotazione sui suoi poli fissi in un moto diurno, mentre il firmamento e il piú alto cielo rimangono immobili.
	
	\item Ciò che ci appare come movimento del Sole non deriva dal suo moto, ma dal moto della Terra e della nostra sfera con la quale ruotiamo attorno al Sole come ogni altro pianeta. La Terra ha, pertanto, piú di un movimento.
	
	\item L'apparente moto retrogrado e diretto dei pianeti non deriva dal loro moto, ma da quello della Terra. Il moto della sola Terra è pertanto sufficiente a spiegare tutte le disuguaglianze che appaiono nel cielo.
	
\end{enumerate}

\subsection{De Revolutionibus}
Nel 1543 viene pubblicato il De Revolutionibus Orbium Caelestium \autocite{copernico:derevolutionibus}. Copernico è in fin di vita e il libro viene pubblicato da Georg Joachim Rheticus, senza il suo consenso, con un commento di Andreas Osiander.

\section{Tycho Brahe}

Tyge Brahe (Tycho Brahe in latino, a volte Ticone in italiano) è nato il 14 dicembre 1546 nel Castello di Knutstorp,  ed è morto a Praga  il 24 ottobre 1601.

È stato matematico e astronomo imperiale corte dell'imperatore del Sacro Romano Impero Rodolfo II d'Asburgo. Le sue osservazioni astronomiche sono le più precise prima dell'avvento del telescopio.
Ammirava il lavoro di Copernico ma era un convinto sostenitore del geocentrismo, propose quindi un modello in cui la Luna e il Sole ruotano attorno alla Terra e gli altri pianeti attorno al Sole.


\section{Keplero}

Johannes von Kepler (Giovanni Keplero in italiano) è nato a Weil der Stadt, Germania, il 27 dicembre 1571 ed è morto a Regensburg (Ratisbona), Germania, il 15 novembre 1630.

È stato un astronomo, astrologo, matematico, musicista e teologo evangelico tedesco. Nel 1599 accetta un incarico quale assistente di Tycho Brahe. Dopo la morte di quest'ultimo diventa matematico e astronomo imperiale. Grazie all'analisi delle osservazioni di Marte fatte da Tycho Brahe scoprì che la sua orbita non è un cerchio ma un'ellisse. Questo lo portò alle tre leggi che portano il suo nome e che regolano il movimento dei pianeti.

\subsection{Astronomia Nova}

Nel 1609 pubblica l'Astronomia Nova, in cui espone le sue prime due leggi

\begin{enumerate}
	\item L'orbita descritta da ogni pianeta nel proprio moto di rivoluzione è un'ellisse di cui il Sole occupa uno dei due fuochi.
	\item Durante il movimento del pianeta, il raggio che unisce il centro del Pianeta al centro del Sole descrive aree uguali in tempi uguali.
\end{enumerate}

\subsection{Harmonicaes Mundi}

Nel 1619 pubblica l'Harmonices Mundi (l'armonia del mondo), nel cui quinto capitolo espone la sua terza legge.

\begin{enumerate}
	\setcounter{enumi}{2}
 	\item I quadrati dei tempi che i pianeti impiegano a percorrere le loro orbite sono proporzionali al cubo delle loro distanze medie dal sole.
 	\[ T^ 2=K\cdot a^3 \]
\end{enumerate}


\section{Galileo}

Galileo Galilei è nato a Pisa il 15 febbraio 1564 ed è morto ad Arcetri l' 8 gennaio 1642.
È stato un fisico, astronomo, filosofo e matematico italiano, ed è considerato il padre della scienza moderna.

%Urbano VIII, nato Maffeo Vincenzo Barberini (Firenze, 5 aprile 1568 – Roma, 29 luglio 1644), è stato il 235º papa della Chiesa cattolica dal 1623 alla morte.


\subsection{Sidereus Nuncius}

Il 12 marzo 1610 pubblica il Sidereus Nuncius \autocite{galileo:nuncius}, un libro in cui descrive alcune osservazioni fatte con il telescopio. Alcune di queste non sono in linea con la teoria Tolemaica:
\begin{itemize}
	\item ci sono montagne sulla Luna, che quindi non è una sfera liscia,
	\item la Via lattea è formata da moltissime stelle,
	\item 4 satelliti orbitano attorno a Giove, i Pianeti Medicei.
\end{itemize}


\subsection{Dialogo sopra i due massimi sistemi del mondo}

Tra il 1624 e il 1630 Galileo scrive il Dialogo sopra i due massimi sistemi del mondo tolemaico e copernicano \autocite{galileo:dialogo}.
Ottiene l'imprimatur\footnote{Imprimatur sta per "Nihil obstat quominus imprimatur" che tradotto letteralmente significa "non esiste alcun impedimento al fatto di essere stampato". La chiesa cattolica autorizzava con questa espressione la stampa di libri. Si tratta di una forma di censura preventiva non obbligatoria in vigore dal 4 maggio 1515} nel 1632.
Il Dialogo non è scritto in latino ma in volgare, come lascia intendere il titolo, è un dialogo fra tre personaggi:

\begin{description}
	\item [Filippo Salviati] (Firenze, 1582 – Barcellona, 22 marzo 1614) è un astronomo e nobile fiorentino, che sostiene la teoria eliocentrica copernicana \autocite{wiki:salviati}.
	\item [Giovanni Francesco Sagredo]  (Venezia, 19 giugno 1571 – Venezia, 1620) è un nobile veneziano amico di Galileo \autocite{wiki:sagredo}. Teoricamente non propende per nessuna delle due teorie ma che ascolta le idee di Salviati e di Simplicio e alla fine preferisce sempre le dimostrazioni a favore della teoria eliocentrica alle tesi di quella geocentrica.
	\item [Simplicio] non è un  personaggio reale, deve il suo nome a Simplicio di Cilicia, un commentatore di Aristotele vissuto nel VI secolo d.C.. Difende la teoria geocentrica.
\end{description}

Il Dialogo viene messo all'indice\footnote{L'Indice dei libri proibiti (Index librorum prohibitorum in latino) era un elenco di pubblicazioni proibite dalla Chiesa cattolica. Fu creato nel 1558 da Paolo IV ed è stato soppresso dopo il Concilio Vaticano II, il 15 novembre 1966, sotto papa Paolo VI} nel 1633.


\subsection{Il processo e l'abiura}

Il 28 settembre 1632 il Sant'Uffizio emette una citazione di comparizione nei confronti di Galileo, che dovrà recarsi a Roma.
Il 22 giugno 1633 viene giudicato colpevole e condannato al carcere.
Abiura subito dopo la condanna.
\begin{quote}
	...con cuor sincero e fede non finta abiuro, maledico e detesto li sudetti errori e eresie...
\end{quote}
Il 1º luglio 1633 gli vengono concessi gli arresti domiciliari, da scontare prima nell'abitazione di Siena dell'amico arcivescovo Antonio Piccolomini, poi nella sua villa di Arcetri.
Si dice che sul letto di morte abbia pronunciato la famosa frase
\begin{quote}
	e pur si muove
\end{quote} 
questa frase indicherebbe come Galileo abbia abiurato per timore dell'in\-qui\-si\-zio\-ne e non per timore di Dio. Non si sa l'abbia pronunciata davvero, non vi sono testimonianze al riguardo.


\subsection{Discorsi}

Nel 1638, dopo il processo Galileo scrisse e pubblicò i Discorsi e dimostrazioni matematiche intorno a due nuove scienze. Non potendoli pubblicare in patria li fece pubblicare a Leida, nei Paesi Bassi.
I Discorsi sono un trattato scientifico riguardante la meccanica.

Allo stesso modo del Dialogo sopra i massimi sistemi, è un dialogo tra Sagredo, Salviati e Simplicio. Si svolge in 4 giornate. 

Durante le prime due giornate i tre discorrono di statica.
Nella prima giornata prende tra l'altro in seria considerazione la possibilità dell'esi\-sten\-za del vuoto, ritenendola una seria ipotesi scientifica. Nel vuoto non  ci sarebbe nessun mezzo in grado di opporre resistenza al movimento, quindi tutti i corpi "discenderebbero con eguale velocità".

Durante la terza e la quarta giornata discorrono invece di dinamica, stabilendo le leggi del moto uniforme, del moto naturalmente accelerato, del moto uniformemente accelerato e delle oscillazioni del pendolo.

Anche i Discorsi sono scritti in volgare.


\pagebreak

\section{Linea del tempo}


\startchronology[startyear=1450,stopyear=1750,height=.2ex]
\chronograduation{50}

\chronoperiode[color=blue, bottomdepth=1cm, topheight=1.3cm]{1473}{1543}{Copernico}

\chronoperiode[color=red, bottomdepth=2.5cm, topheight=2.8cm]{1483}{1546}{Lutero}

\chronoperiode[color=blue, bottomdepth=4cm, topheight=4.3cm]{1546}{1601}{Brahe}

\chronoperiode[color=red, bottomdepth=5.5cm, topheight=5.8cm]{1548}{1600}{Bruno}

\chronoperiode[color=blue, bottomdepth=7cm, topheight=7.3cm]{1564}{1642}{Galileo}

\chronoperiode[color=red, bottomdepth=1cm, topheight=1.3cm]{1568}{1644}{Urbano VII}

\chronoperiode[color=blue, bottomdepth=2.5cm, topheight=2.8cm]{1571}{1630}{Keplero}

\chronoperiode[color=cyan, bottomdepth=4cm, topheight=4.3cm]{1625}{1712}{Cassini}

\chronoperiode[color=cyan, bottomdepth=5.5cm, topheight=5.8cm]{1643}{1727}{Newton}

\chronoperiode[color=cyan, bottomdepth=2cm, topheight=2.3cm]{1646}{1716}{Leibniz}

\chronoevent[markdepth=8.6cm, conversionmonth=false]{1638}{Discorsi}

\chronoevent[markdepth=7.4cm, conversionmonth=false]{1609}{Astronomia Nova}

\chronoevent[markdepth=6.2cm, conversionmonth=false]{1632}{Dialogo}

\chronoevent[markdepth=5cm, conversionmonth=false]{1582}{Riforma Gregoriana}

\chronoevent[markdepth=5cm, conversionmonth=false]{1703}{Arithmetique Binaire}

\chronoevent[markdepth=4cm, conversionmonth=false]{1619}{Harmonices Mundi}

\chronoevent[markdepth=3cm, conversionmonth=false]{1543}{De Revolutionibus}

\chronoevent[markdepth=3cm, conversionmonth=false]{1687}{Principia} % Philosophiae Naturalis Principia Mathematica 

\chronoevent[markdepth=2cm, conversionmonth=false]{1610}{Sidereus Nuncius}

\chronoevent[markdepth=1cm, conversionmonth=false]{1671}{Calcolatrice a tamburo}

\chronoevent[markdepth=1cm, conversionmonth=false]{1508}{Commentariolus}

\stopchronology

% blue, red, cyan, purple and yellow,

\pagebreak

\printbibliography

\end{document}
